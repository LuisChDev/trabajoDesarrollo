\documentclass[twocolumn]{article}
\author{
  Chavarriaga, Luis\\
  \texttt{lchavarriaga@utb.edu.co}
  \and
  Agames, Juan\\
  \texttt{Jagamez321@gmail.com}
  \and
  Pájaro, Juan\\
  \texttt{juanpablo192712@hotmail.com}
  \and
  Cano, Luigi\\
  \texttt{napellido@utb.edu.co}
}
\title{Implicaciones legales, sociales y económicas del desarrollo tecnológico
  en Colombia}
\date{Abril 2021}

\usepackage[utf8]{inputenc}
\usepackage[T1]{fontenc}

\usepackage{biblatex}
\addbibresource{Biblio.bib}

\begin{document}
\maketitle

\section{Introducción}
La investigación se constituye en uno de los pilares para el crecimiento de una
sociedad. En particular, el desarrollo tecnológico es, en pleno siglo XXI, el
motor del nuevo sector cuaternario y la base para la expansión de los otros
sectores. Por ello resulta importante entender cómo los factores externos
afectan el crecimiento del mismo, y principal entre ellos es el marco normativo
del país correspondiente.

En este documento, se explora la situación actual de la investigación en el
ámbito de la República de Colombia, el marco legal vigente, las implicaciones
éticas de la responsabilidad del ingeniero y su impacto en la sociedad.

\section{El gobierno}

\section{La ética del ingeniero}

¿Qué ha ocurrido con la ingeniería de sistemas en Colombia? En 2002, Richard
Riaño Botina afirmó descubrir los nexos entre 255 funcionarios del gobierno y
las fuerzas paramilitares. El `Hacker de la fiscalía', como se le ha denominado,
dice ser experto en seguridad informática y que recuperó esta información de
documentos de esta entidad gubernamental, según se ha dicho, de manera ilegal
\cite{noauthor_quien_2020}. En 2014, otro evento sacudió al país de la misma
forma. Andrés Sepúlveda, de la campaña de Óscar Iván Zuluaga, afirmó tener
posesión de documentos secretos que desacreditaban el proceso de paz en La
Habana. Aunque al final resultó que el `hacker' Sepúlveda recurrió a medios no
técnicos para adquirir esta información (entre otros, sobornar a funcionarios
públicos) \cite{semanacom_andres_nodate}, ambas historias sirven para poner de
manifiesto la posición ante la que encuentran los profesionales que trabajan con
información potencialmente confidencial. La ingeniería de sistemas, abarcando
campos como `big data' (análisis estadístico), las redes sociales y las bases de
datos, es el campo profesional que más relevancia tiene ante los problemas
éticos que surgen.




\section{Impacto de la tecnología}

\section{Conclusiones}

\printbibliography{}

\end{document}
% Local Variables:
% ispell-local-dictionary: "castellano"
% End:
